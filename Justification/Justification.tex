\chapter{Justificación}

El estudio del interior de grandes estructuras, en especial las geológicas, puede ser realizado a través de muones atmosféricos producidos por rayos cósmicos. El flujo de muones a nivel del mar es $\approx$ 1  muon/cm$^2$min y es capaz de penetrar varios kilómetros de roca \cite{Ariga2018}. La radiografía de muones o muongrafía en estructuras geológicas se hace por medio de un telescopio de muones el cual registra del flujo de muones que atraviesan la estructura en diferentes direcciones. La atenuación del flujo de muones provee información de la distribución de densidad dependiendo de la cantidad del material atravesado.\\

La técnica de la muongrafía fue propuesta por George en 1955 \cite{George1955} e implementada inicialmente por Alvarez en 1970 \cite{Alvarez1970}. Actualmente se utiliza en diversas áreas, siendo la geología la más representativa. En este campo se han realizado trabajos de muongrafía de volcanes \cite{Tanaka2009, Lesparre2012, Carbone2013}, cavernas \cite{Saracino2017, Olh2013}, acuíferos \cite{Jourde2016}, reservorios de CO2 \cite{Zhong2015, Klinger2015, Zhong2016} y glaciares \cite{Nishiyama2017, Ariga2018}.\\

En años recientes los esfuerzos se han centrado en el refinamiento de la técnica para estudiar estructuras geológicas, abordando temas como el ruido de fondo generado por los muones de baja energía \cite{Nishiyama2014,Gomez2017}, el ruido causado por la componente electromagnética de la lluvias aéreas de partículas \cite{KUSAGAYA2015, Nishiyama2014Noise, Marteau2012Noise}, los efectos de la composición de la roca sobre la muongrafía \cite{Lechmann2018}, el aumento de su eficiencia mediante sistemas ToF \cite{Shi2014} y su viabilidad para hacer estudios del comportamiento dinámico de estructuras geológicas\cite{Jourde2016}.\\

Un telescopio de muones que reduzca el ruido fondo debe tener tres características:

\begin{itemize}
    \item Identificar y filtrar los muones de baja energía ($<$ 1 GeV) causantes de la mayor parte de ruido de fondo. 
    \item Identificar los $e^{\pm}$ y $\gamma$ generados por lluvias aéreas que pueden emular la trayectoria que trazaría un muón proveniente del volcán. 
    \item Identificar y filtrar los muones que atraviesan el telescopio desde la parte trasera del detector.
\end{itemize}

La eliminación de eventos falsos mejora la estimación de la densidad de la estructura escaneada a partir del flujo de muones emergente.\\

En este proyecto se propone el desarrollo y calibración de un telescopio de muones que estime las diferencias de densidades de una estructura teniendo en cuenta las fuentes de ruido anteriormente expuestas.





















