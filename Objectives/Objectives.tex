\chapter{Objetivos}
\section{\textbf{Objetivo General}}

Diseñar, construir y calibrar el sistema electrónico de un telescopio de muones para el estudio de estructuras geológicas capaz de filtrar las principales fuentes de ruido encontradas en la muongrafía hasta la fecha.

\section{\textbf{Objetivos Específicos}}

\begin{itemize}
    \item Diseñar y calibrar el sistema electrónico de un hodoscopio que permita registrar el flujo de muones en diferentes direcciones.
    
    \item  Implementar un sistema ToF de alta resolución temporal capaz de discriminar los muones con momento $< 1$ GeV y los muones de albedo.

    \item  Diseñar y calibrar el sistema electrónico de un subdetector (WCD) que ayude a diferenciar los $\mu^{\pm}$ de los $e^{\pm}$ y $\gamma$ generados por lluvias aéreas extendidas (EAS).
    
    %\item Calibrar el hodoscopio con el fin de obtener una respuesta uniforme en cada uno de sus píxeles y obtener una reconstrucción confiable del flujo de muones.
    
    %\item Calibrar el WCD para obtener el punto óptimo de operación que maximice la discriminación entre  $\mu^{\pm}$ y $e^{\pm}, \gamma$.
    
    %\item Acoplar el hodoscopio y el WCD en un sistema conjunto de adquisición que sea robusto e independiente.
    
    \item Validar el desempeño del Telescopio de Muones (MuTe) en la reducción del ruido en muografía. 
    
    %\item Instalar sistemas periféricos que suministren información acerca del funcionamiento del detector.
    
    %\item Hacer un análisis detallado de las principales fuentes de ruido en la muongrafía a partir de los datos recolectados.
    
\end{itemize}