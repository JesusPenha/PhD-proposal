\chapter{Planteamiento del problema}

La muongrafía es una técnica no-invasiva que se utiliza para estudiar grandes estructuras antrópicas o naturales. Sus aplicaciones van desde detección de materiales ocultos en contenedores \cite{Blanpied2015}, arqueología \cite{Morishima2017, Gmez2016, Alvarez1970}, exploración geológica en Marte \cite{Kedar2013}, inspección de plantas nucleares \cite{Fujii2013}, cavidades subterráneas \cite{Saracino2017} y  vulcanología \cite{Tanaka2005, Tanaka2009, Lesparre2010, Lesparre2011, Lesparre2012}.\\

Actualmente las herramientas más utilizadas en el estudio de volcanes son la sismología, gravimetría y tomografía eléctrica. Estas tienen una resolución espacial y capacidad de penetración baja \cite{RosasCarbajal2017}, además los datos deben ser registrados de manera directa sobre la superficie del volcán, \cite{Marteau2012}. La muongrafía ofrece una alternativa con resolución espacial de decenas de metros \cite{Lesparre2012}, una gran capacidad de penetración y la adquisición de información es no-invasiva \cite{Nishiyama2014}.\\

Su funcionamiento se basa en la medición del flujo de muones que cruzan la estructura en diferentes direcciones. Esta medición se hace a través de un hodoscopio\footnote{Instrumento que mide la trayectoria de partículas cargadas haciendo uso de dos o más planos de detección}. Las diferencias de flujo que se proyectan sobre el elemento sensible permiten extraer información de la densidad interna del objeto escaneado.\\

Sin embargo, esta metodología presenta algunos problemas como la sub-estimación de la densidad debido al registro de eventos falso-positivos, los cuales se generan por tres fuentes: los muones horizontales que inciden desde la parte trasera del detector, los muones de baja energía que son dispersados por la superficie del volcán y partículas cargadas procedentes de lluvias aéreas extendidas (EAS) \cite{Nishiyama2014,Gomez2017}.\\

Para reducir estos efectos se han desarrollado diversas técnicas basadas en: la implementación de sistemas de medición del tiempo de vuelo (ToF) para eliminar los muones de albedo\footnote{Muones atmosféricos reflejados hacia la atmósfera por la tierra o generados en ella.)} \cite{Marteau2014, Cimmino2017}, la instalación de paneles absorbentes para filtrar los muones de baja energía y el aumento de la cantidad de paneles sensibles para disminuir la probabilidad de detectar eventos generados por EAS \cite{Lesparre2012}.\\

La instalación de paneles (absorbentes o sensibles) repercute en el aumento de la complejidad del detector. La mejor opción para la eliminación del ruido de fondo son los sistemas ToF y de identificación de partículas. Actialmente, los telescopios MuRay y Diaphane tienen sistemas ToF con resolución de 400 y 240 ps respectivamente \cite{Cimmino2017, Marteau2014}, sin embargo no es suficiente para discriminar los muones y electrones de baja energía.\\

En este trabajo se propone la implementación de un telescopio de muones híbrido capaz de reducir las principales fuentes de ruido que pueden afectar la muongrafía de estructuras volcánicas.

% La principal fuente de ruido en muongrafía son electrones y muones con momento $<$ 1 GeV/c como se reporta en \cite{Nishiyama2014,Gomez2017}.